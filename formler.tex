\documentclass{article}

\usepackage[margin=1.2in]{geometry} % page margins
\usepackage[utf8]{inputenc}         % UTF-8 charset fix
\usepackage{parskip}                % paragraph vertical margin

\renewcommand{\thesection}{}                       % remove subsection prefix
\renewcommand{\thesubsection}{\arabic{subsection}} % same as above

\begin{document}

\section*{Formler sannstat}

1. Definiera begreppen sannolikhet, fördelningsfunktion...
- Sannolikhet: ett mått på hur troligt det är att en viss händelse inträffar.
- Fördelningsfunktion: ett begrepp som betecknar ett uttryck för hur sannolika
olika utfall i ett utfallsrum är.
- Täthets- eller sannolikhetsfunktion: En bild av hur sannolika olika resultat
är i förhållande till varandra till skillnad från fördelningsfunktionen som ger
sannolikheten att variabeln antar värden som "ligger till vänster" om en given
punkt x på talaxeln.

Fördelningsfunktionen är integralen till täthetsfunktionen.

Stora talens lag
Låt $X_1, X_2, ...$ vara oberoende och likafördelade stokastiska variabler, var
och en med väntevärdet $\mu$ , och sätt
$$
X_n = \sum\limits_{i=1}^{n}\frac{X_i}{n}.
$$
Då gäller, för alla $\epsilon > 0$, att
$$
P(\mu - \varepsilon \le X_n \le \mu + \varepsilon) \implies 1, n \implies \infty.
$$

\subsection{Sannolikteorins grunder}
% \begin{itemize}
% \end{itemize}
\end{document}
