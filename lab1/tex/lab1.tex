\documentclass{article}

\usepackage[margin=1.2in]{geometry} % page margins
\usepackage[utf8]{inputenc}         % UTF-8 charset fix
\usepackage{mathtools}              % do math
\usepackage{parskip}                % paragraph vertical margin

\renewcommand \thesubsection{\arabic{section}.\Alph{subsection}}

\begin{document}

\section*{Förberedelseuppgifter}

\subsection*{1. Sannolikhet}

Ett mått på hur troligt det är att en viss händelse inträffar.

\subsection*{Täthetsfunktion}

En bild av hur sannolika olika resultat är i förhållande till varandra till
skillnad från fördelningsfunktionen som ger sannolikheten att variabeln antar
värden som "ligger till vänster" om en given punkt x på talaxeln.

\subsection*{Fördelningsfunktion}

Ett begrepp som betecknar ett uttryck för hur sannolika olika utfall i ett
utfallsrum är.

\subsection*{Samband mellan fördelnings- och täthetsfunktion}

Fördelningsfunktionen är integralen till täthetsfunktionen.

\subsection*{2a. Närmevärde till lambda}
Vi integrerar täthetsfunktionen från 1 till 10 och vet att = 1.
$\lambda = 0.4267$
\subsection*{2b. Fördelningsfunktion för X}
$F = \lambda(\log(x)-\lambda e^{\frac{-x}{\lambda}})$
\subsection*{2c. Sannolikhet X mindre än 7}
$\lambda = 0.4267$ \\
$\int \limits_1^{7} f_X(x)  \, \mathrm{d}x = 0.847796$
\subsection*{2d. E(X)}
$E(X) = \int \limits_1^{10} xf_X(x)  \, \mathrm{d}x = 3.86523$

\subsection*{3a}
$
E[cos(U)] =
\frac{1}{2\pi} \int \limits_0^{2\pi} cos(x) \, \mathrm{d}x =
\left[sin(x)\right]_0^{2\pi} =
0
$

\subsection*{3b}

$
E[sin(U)^2] =
\frac{1}{2\pi} \int \limits_0^{2\pi} sin(x)^2 \, \mathrm{d}x =
\left[\frac{x-sin(x)cos(x)}{2}\right]_0^{2\pi} =
\frac{1}{2}
$

\subsection*{4. Stora talens lag}

Låt $ X_1, X_2, ..., X_n $ vara oberoende och likafördelade stokastiska
variabler, var och en med väntevärdet $\mu$, och sätt:

$
X_n = \sum\limits_{i=1}^{n}\frac{X_i}{n}.
$

Då gäller, för alla $\varepsilon > 0$, att:

$
P(\mu - \varepsilon \le X_n \le \mu + \varepsilon) \to 1 \, , \, n \to \infty
$
\subsection*{5. Centrala gränsvärdessatsen}

Om man tittar på tillräckligt många oberoende s.v. med ett gemensamt väntevärde
kommer dess fördelning efterlikna en normalfördelning ty medelvärdet av alla
s.v. kommer gå mot väntevärdet.

\section*{Problem 0}
\subsection*{1}
   
  $ P(X_1 \le 10) = 1.5556e-06 $      \\       
  $ P(X_2 \le 10) = 0 $               \\
  $ P(X_3 \le 10) = 9.9999e-01 $\\
  \\
\subsection*{2}
  $ P(X_1 > 15) = 9.9960e-01 $\\
  $ P(X_2 > 15) = 1.0000e+00 $\\
  $ P(X_3 > 15) = 5.7777e-08 $\\
  \\
\subsection*{3}
  $ P(7 \le X_1 \le 15) = 1.5390e-06 $\\ 
  $ P(7 \le X_1 \le 15) = 0 $ \\
  $ P(7 \le X_1 \le 15) = 4.0400e-04 $
   
\section*{Problem 1}
Vi ser att vår funktion kan approximeras med den exponentiala fördelningen. Vår
funktion är vidare bara $>$ 0 i intervallet 1 till 10.
\\
\section*{Problem 2}
Vi ser att my ändrar väntevärdet(flyttar grafen), sigma påverkar
standardavikelsen(ändrar utseendet) och rå ändrar relationen mellan x- och
y-led
\\
\section*{Problem 3}
Slumpen gör att vissa resultat skiljer sig ifrån den röda linjen. I och med att
talen slumpas utifrån en exponential fördelning så medför det att E(slumptalen)
= E(exponentialfördelningen), samma för variansen.

\section*{Problem 4}
\subsection*{Problem 4a}
Det ser ut som förväntat eftersom teorin är att när antalet försök går mot
oändligheten kommer medelvärdet konvergera mot väntevärdet(i detta fall 0.5).
\subsection*{Problem 4b}
N är antalet oberoende stokastiska variabler. 
\subsection*{Problem 4c}
Det kommer gå mot en normalfördelning (def centrala gränsvärdessatsen)
\subsection*{Problem 4d}
Grafen skjuts linjärt i X-led(med ett medel i N * mu), grafens symmetriska form
uppstår vid N = 35.
\subsection*{Problem 4e}
De är normalfördelade enligt centrala gränsvärdessatsen. 

Om vi plockar ut n tal ur en mängd med väntevärdet my så kan summan av alla
dessa tal approximeras av en normalfördelning med mittpunkten i n*my.  
\subsection*{Problem 4f}
Vi får svar 3.3e15 typ ish. Det är orimligt och ser när vi plottar fördelningen
att vi har ett jättehopp där värdet avviker markant. Medianen ger ett bättre
väntevärde som blir ca 4.
\subsection*{Problem 5}
Hämtar n\_sum stycken slumpade heltal från totalNoSamples. \\sampleDraws blir
dessa index från X. \\ sampleDraws = X(randsample(totalNoSamples, n\_sum, 1)) \\j
"pelare", summerar alla tal i sampleDraws på det indexet. \\ yBoot(j) =
sum(sampleDraws), "fyller i glapp"
\\
Bootstrap: Utvidgar en liten mängd spridd mätdata och få en bra approximation av
hur datat hade sett ut ifall man hade haft fler mätvärden. Om man ökar antalet
bootstrapreplikat så blir approximationen bättre.
\\Om man ökar my så förskjuts väntevärdet i x-led.
\\M är antalet indata, så högre M gör att det blir färre "glapp" att fylla i. Så
om M ökar blir resultatet bättre för initiala mätdata. Stort M gör att man inte
behöver göra någon bootstrap.
\end{document}
