\documentclass{article}

\usepackage[margin=1.2in]{geometry} % page margins
\usepackage[utf8]{inputenc}         % UTF-8 charset fix
\usepackage{mathtools}              % do math
\usepackage{parskip}                % paragraph vertical margin

\renewcommand \thesubsection{\arabic{section}.\Alph{subsection}}

\begin{document}

\section*{Förberedelseuppgifter}

\addtocounter{section}{1}

\subsection{Sannolikhet}

Ett mått på hur troligt det är att en viss händelse inträffar.

\subsection{Täthetsfunktion}

En bild av hur sannolika olika resultat är i förhållande till varandra till
skillnad från fördelningsfunktionen som ger sannolikheten att variabeln antar
värden som "ligger till vänster" om en given punkt x på talaxeln.

\subsection{Fördelningsfunktion}

Ett begrepp som betecknar ett uttryck för hur sannolika olika utfall i ett
utfallsrum är.

\subsection{Samband mellan fördelnings- och täthetsfunktion}

Fördelningsfunktionen är integralen till täthetsfunktionen.

\addtocounter{section}{1}
\usecounter{subsection}

\subsection{TODO}
\subsection{TODO}
\subsection{TODO}
\subsection{TODO}

\addtocounter{section}{1}
\usecounter{subsection}

\subsection{}

$
E[cos(U)] =
\frac{1}{2\pi} \int \limits_0^{2\pi} cos(x) \, \mathrm{d}x =
\left[sin(x)\right]_0^{2\pi} =
0
$

\subsection{}

$
E[sin(U)^2] =
\frac{1}{2\pi} \int \limits_0^{2\pi} sin(x)^2 \, \mathrm{d}x =
\left[\frac{x-sin(x)cos(x)}{2}\right]_0^{2\pi} =
\frac{1}{2}
$

\addtocounter{section}{1}
\usecounter{subsection}

\newpage

\subsection{Stora talens lag}

Låt $ X_1, X_2, ..., X_n $ vara oberoende och likafördelade stokastiska
variabler, var och en med väntevärdet $\mu$, och sätt:

$
X_n = \sum\limits_{i=1}^{n}\frac{X_i}{n}.
$

Då gäller, för alla $\varepsilon > 0$, att:

$
P(\mu - \varepsilon \le X_n \le \mu + \varepsilon) \to 1 \, , \, n \to \infty
$

\addtocounter{section}{1}
\usecounter{subsection}

\subsection{Centrala gränsvärdessatsen}

Innebär att om man adderar ett stort antal oberoende slumpmässiga variabler,
eventuellt med olika sannolikhetsfördelningar, men med ändliga varianser, kommer
summan att gå mot en normalfördelning.

\end{document}
